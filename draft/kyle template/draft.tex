\documentclass{article}

\input{preamble}
\usepackage{tikz-3dplot}
\usepackage{booktabs}
\usepackage{titling}
\usepackage{fancyhdr}
\usetikzlibrary{patterns}

\usepackage{tcolorbox}
\usepackage[export]{adjustbox}
\usepackage{tikz}

\tcbset{colback=blue!7!white}
\tcbsetforeverylayer{colframe=blue!75!black}


\pretitle{\vspace{-4em}\begin{flushleft}\LARGE} % Adjust the size and alignment
    \posttitle{\end{flushleft}}
    \preauthor{\begin{flushleft}\large} % Adjust the size and alignment
    \postauthor{\hspace{2em} \large \thedate\end{flushleft}} % Place author and date on the same line
    \predate{} % Remove the default date formatting
    \postdate{} % Remove the default date formatting

\pagestyle{fancy}
\fancyhf{} % Clear all header and footer fields
\fancyfoot[R]{\thepage} % Right-align the page number in the footer
\renewcommand{\headrulewidth}{0pt} % Remove the header rule
\renewcommand{\footrulewidth}{0pt} % Remove the footer rule
\setlength{\fboxrule}{1pt}

\geometry{
    top=2cm,    % Top margin
    bottom=3cm, % Bottom margin
    left=2.5cm, % Left margin
    right=2.5cm % Right margin
}
\title{\bfseries CHEG304 Honors: The Best Seat at UD}
\author{Maxmillian Stratton, Quinlan Kraft, Jackson Rau, Braden Rogers, Kyle Wodehouse}
\date{}

\begin{document}
\maketitle

\section{Abstract}

abstract shall be completed once the rest of the writing is done

\section{Introduction and Background}

\section{Methods}
\indent

In an effort to collect the most accurate data, it’s important to adhere to the same procedure during each instance of data collection. To start, identify the building, room number, and room type that the measurements are being taken in. Next, identify the desk “type” that is being measured. When performing this step, each time you encounter a new desk “type”, add it to a reference document with an image and description to catalog all different styles of desk encountered. Doing so allows for easier data collection as descriptions and images don’t have to be repeated between rooms when the desks are of the same “type”, and can instead be generalized per desk “type”. 

Next, begin taking physical measurements of the desk. Despite the generalization for descriptions and images, it’s important to take separate data measurements per room. Even when the desks appear visually similar or the same, they could have slightly different measurements. Start by measuring the height from the floor to the tallest point of the seat of the desk (i.e. where someone would sit). Next, measure the height from the floor to the tallest point of the physical desk (i.e. the writing surface). For both of these measurements, the seat and desk surface may not be flat and/or could be at an angle, so it’s important to measure to the highest point of the surface to be consistent. Lastly, measure the width and length of the “useable” desk area (i.e. not including the armrest portion) by approximating it as a rectangle. For shared tables or desks, this can be calculated by finding the overall area and then dividing by the number of seats at the table, thus giving the usable area per person at the table. It is recommended to collect this data in a spreadsheet application with the following columns: building, room number, room type, desk type, seat height, desk height, desk width, and desk length.


\section{Results and Discussion}

\section{Conclusions}

\section{References}

\section{Appendix}


\end{document}